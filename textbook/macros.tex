%%%%%%%%%%%%%%%%%%%%%%%%%%%%%%%%%%%%%%%%%%%%%%%%%%%%%%%%%
%                tufte-book style
%%%%%%%%%%%%%%%%%%%%%%%%%%%%%%%%%%%%%%%%%%%%%%%%%%%%%%%%%

%%
% If they're installed, use Bergamo and Chantilly from www.fontsite.com.
% They're clones of Bembo and Gill Sans, respectively.
%\IfFileExists{bergamo.sty}{\usepackage[osf]{bergamo}}{}% Bembo
%\IfFileExists{chantill.sty}{\usepackage{chantill}}{}% Gill Sans

%\usepackage{microtype}

%%
% Just some sample text
\usepackage{lipsum}

%%
% For nicely typeset tabular material
\usepackage{booktabs}

%%
% For graphics / images
\usepackage[pdftex]{graphicx}
\setkeys{Gin}{width=\linewidth,totalheight=\textheight,keepaspectratio}
  \graphicspath{{./figures/}}
  \DeclareGraphicsExtensions{.pdf,.jpg,.png}

% The fancyvrb package lets us customize the formatting of verbatim
% environments.  We use a slightly smaller font.
\usepackage{fancyvrb}
\fvset{fontsize=\normalsize}

%%
% Prints argument within hanging parentheses (i.e., parentheses that take
% up no horizontal space).  Useful in tabular environments.
\newcommand{\hangp}[1]{\makebox[0pt][r]{(}#1\makebox[0pt][l]{)}}

%%
% Prints an asterisk that takes up no horizontal space.
% Useful in tabular environments.
\newcommand{\hangstar}{\makebox[0pt][l]{*}}

%%
% Prints a trailing space in a smart way.
\usepackage{xspace}

%%
% Some shortcuts for Tufte's book titles.  The lowercase commands will
% produce the initials of the book title in italics.  The all-caps commands
% will print out the full title of the book in italics.
\newcommand{\vdqi}{\textit{VDQI}\xspace}
\newcommand{\ei}{\textit{EI}\xspace}
\newcommand{\ve}{\textit{VE}\xspace}
\newcommand{\be}{\textit{BE}\xspace}
\newcommand{\VDQI}{\textit{The Visual Display of Quantitative Information}\xspace}
\newcommand{\EI}{\textit{Envisioning Information}\xspace}
\newcommand{\VE}{\textit{Visual Explanations}\xspace}
\newcommand{\BE}{\textit{Beautiful Evidence}\xspace}

\newcommand{\TL}{Tufte-\LaTeX\xspace}

% Prints the month name (e.g., January) and the year (e.g., 2008)
\newcommand{\monthyear}{%
  \ifcase\month\or January\or February\or March\or April\or May\or June\or
  July\or August\or September\or October\or November\or
  December\fi\space\number\year
}


% Prints an epigraph and speaker in sans serif, all-caps type.
\newcommand{\openepigraph}[2]{%
  %\sffamily\fontsize{14}{16}\selectfont
  \begin{fullwidth}
  \sffamily\large
  \begin{doublespace}
  \noindent\allcaps{#1}\\% epigraph
  \noindent\allcaps{#2}% author
  \end{doublespace}
  \end{fullwidth}
}

% Inserts a blank page
\newcommand{\blankpage}{\newpage\hbox{}\thispagestyle{empty}\newpage}

\usepackage{units}

% Typesets the font size, leading, and measure in the form of 10/12x26 pc.
\newcommand{\measure}[3]{#1/#2$\times$\unit[#3]{pc}}

% Macros for typesetting the documentation
\newcommand{\hlred}[1]{\textcolor{Maroon}{#1}}% prints in red
\newcommand{\hangleft}[1]{\makebox[0pt][r]{#1}}
\newcommand{\hairsp}{\hspace{1pt}}% hair space
\newcommand{\hquad}{\hskip0.5em\relax}% half quad space
\newcommand{\TODO}{\textcolor{red}{\bf TODO!}\xspace}
\newcommand{\na}{\quad--}% used in tables for N/A cells
\providecommand{\XeLaTeX}{X\lower.5ex\hbox{\kern-0.15em\reflectbox{E}}\kern-0.1em\LaTeX}
\newcommand{\tXeLaTeX}{\XeLaTeX\index{XeLaTeX@\protect\XeLaTeX}}
% \index{\texttt{\textbackslash xyz}@\hangleft{\texttt{\textbackslash}}\texttt{xyz}}
\newcommand{\tuftebs}{\symbol{'134}}% a backslash in tt type in OT1/T1
\newcommand{\doccmdnoindex}[2][]{\texttt{\tuftebs#2}}% command name -- adds backslash automatically (and doesn't add cmd to the index)
\newcommand{\doccmddef}[2][]{%
  \hlred{\texttt{\tuftebs#2}}\label{cmd:#2}%
  \ifthenelse{\isempty{#1}}%
    {% add the command to the index
      \index{#2 command@\protect\hangleft{\texttt{\tuftebs}}\texttt{#2}}% command name
    }%
    {% add the command and package to the index
      \index{#2 command@\protect\hangleft{\texttt{\tuftebs}}\texttt{#2} (\texttt{#1} package)}% command name
      \index{#1 package@\texttt{#1} package}\index{packages!#1@\texttt{#1}}% package name
    }%
}% command name -- adds backslash automatically
\newcommand{\doccmd}[2][]{%
  \texttt{\tuftebs#2}%
  \ifthenelse{\isempty{#1}}%
    {% add the command to the index
      \index{#2 command@\protect\hangleft{\texttt{\tuftebs}}\texttt{#2}}% command name
    }%
    {% add the command and package to the index
      \index{#2 command@\protect\hangleft{\texttt{\tuftebs}}\texttt{#2} (\texttt{#1} package)}% command name
      \index{#1 package@\texttt{#1} package}\index{packages!#1@\texttt{#1}}% package name
    }%
}% command name -- adds backslash automatically
\newcommand{\docopt}[1]{\ensuremath{\langle}\textrm{\textit{#1}}\ensuremath{\rangle}}% optional command argument
\newcommand{\docarg}[1]{\textrm{\textit{#1}}}% (required) command argument
\newenvironment{docspec}{\begin{quotation}\ttfamily\parskip0pt\parindent0pt\ignorespaces}{\end{quotation}}% command specification environment
\newcommand{\docenv}[1]{\texttt{#1}\index{#1 environment@\texttt{#1} environment}\index{environments!#1@\texttt{#1}}}% environment name
\newcommand{\docenvdef}[1]{\hlred{\texttt{#1}}\label{env:#1}\index{#1 environment@\texttt{#1} environment}\index{environments!#1@\texttt{#1}}}% environment name
\newcommand{\docpkg}[1]{\texttt{#1}\index{#1 package@\texttt{#1} package}\index{packages!#1@\texttt{#1}}}% package name
\newcommand{\doccls}[1]{\texttt{#1}}% document class name
\newcommand{\docclsopt}[1]{\texttt{#1}\index{#1 class option@\texttt{#1} class option}\index{class options!#1@\texttt{#1}}}% document class option name
\newcommand{\docclsoptdef}[1]{\hlred{\texttt{#1}}\label{clsopt:#1}\index{#1 class option@\texttt{#1} class option}\index{class options!#1@\texttt{#1}}}% document class option name defined
\newcommand{\docmsg}[2]{\bigskip\begin{fullwidth}\noindent\ttfamily#1\end{fullwidth}\medskip\par\noindent#2}
\newcommand{\docfilehook}[2]{\texttt{#1}\index{file hooks!#2}\index{#1@\texttt{#1}}}
\newcommand{\doccounter}[1]{\texttt{#1}\index{#1 counter@\texttt{#1} counter}}



%%%%%%%%%%%%%%%%%%%%%%%%%%%%%%%%%%%%%%%%%%%%%%%%%%%%%%%%%
%                Packages
%%%%%%%%%%%%%%%%%%%%%%%%%%%%%%%%%%%%%%%%%%%%%%%%%%%%%%%%%
\usepackage{amsmath,amssymb,amstext,amsfonts}

\usepackage[T1]{fontenc}
\usepackage[latin9]{inputenc}
\usepackage{lmodern}
\usepackage{latexsym}
%\usepackage[pdftex]{graphicx}
%  \graphicspath{{./figures/}}
%  \DeclareGraphicsExtensions{.pdf,.jpg,.png}
\usepackage{color}
\usepackage{subfigmat}
%\usepackage{keyval,times}
\usepackage{fancyhdr}
	\pagestyle{fancy}
	\fancyhead[LE,RO]{} % make is so nothing appears in header on left for even and right for odd pages
\usepackage{theorem}
\usepackage{nicefrac}
\usepackage{listings}  % for code
\usepackage[small,bf]{caption}  % for better figure captions
\usepackage{makeidx}  % for making an index
\usepackage{longtable}
\usepackage{units}
\usepackage{algorithmic,algorithm}
\usepackage{hyperref}
\hypersetup{%
   pdftitle={quadrotorbook},
   pdfauthor={R. Beard, T. McLain},
   pdfkeywords={Multi-rotors, attitude estimation, control.},
   bookmarksnumbered,
   pdfstartview={FitH},
   colorlinks=true,
   breaklinks=true,
   citecolor=blue,
   linkcolor=blue,
}%
\usepackage{pdfpages}
%\usepackage{tocloft}  % change spacing in table of contents
%	\setlength\cftparskip{-2pt}
%	\setlength\cftbeforechapskip{0pt}
\usepackage{cancel} % cancel marks
\usepackage{fancybox}  % oval boxes in tables

% Generates the index
\usepackage{makeidx}
\makeindex

\setcounter{secnumdepth}{3}  % defines which sections are numbered


%%%%%%%%%%%%%%%%%%%%%%%%%%%%%%%%%%%%%%%%%%%%%%%%%%%%%%%%%
%                Macros
%%%%%%%%%%%%%%%%%%%%%%%%%%%%%%%%%%%%%%%%%%%%%%%%%%%%%%%%%
\newcommand{\OMIT}[1]{{}}
%\newcommand{\TODO}[1]{{\color{green} To do: #1}}
\newcommand{\rwbcomment}[1]{{\color{red} RWB: #1}}
\newcommand{\twmcomment}[1]{{\color{blue} TWM: #1}}
\newcommand{\jbwcomment}[1]{{\color{red} JBW: #1}}
\newcommand{\pde}[2]{\frac{\partial#1}{\partial#2}}
\newcommand{\trace}[1]{\text{tr}\left(#1\right)}
\newcommand{\rank}[1]{\text{rank}\left(#1\right)}
\renewcommand{\ss}[1]{\left\lfloor#1\right\rfloor_{\times}}
%\renewcommand{\ss}[1]{{#1^\wedge}}
\newcommand{\cov}{\text{\em cov}}
\newcommand{\var}{\text{\em var}}
\newcommand{\stdev}{\text{\em stdev}}
\newcommand{\defeq}{\stackrel{\triangle}{=}}
\newcommand{\norm}[1]{\left\|#1\right\|}
\newcommand{\abs}[1]{\left|#1\right|}
\newcommand{\re}{{\mathbb{R}}}
\newcommand{\tr}{\top}
\newcommand{\half}{\frac{1}{2}}
%\newcommand{\Vair}{V_{\text{air}}}
%\newcommand{\Vgrd}{V_{\text{grd}}}
\newcommand{\mass}{\mathsf{m}}
\newcommand{\Acal}{{\mathcal{A}}}
\newcommand{\Ccal}{{\mathcal{C}}}
\newcommand{\Gcal}{{\mathcal{G}}}
\newcommand{\Hcal}{{\mathcal{H}}}
\newcommand{\Jcal}{{\mathcal{J}}}
\newcommand{\Lcal}{{\mathcal{L}}}
\newcommand{\Rcal}{{\mathcal{R}}}
\newcommand{\Scal}{{\mathcal{S}}}
\newcommand{\Ucal}{{\mathcal{U}}}
\newcommand{\Xcal}{\mathcal{X}}
\newcommand{\abf}{\mathbf{a}}
\newcommand{\bbf}{\mathbf{b}}
\newcommand{\cbf}{\mathbf{c}}
\newcommand{\dbf}{\mathbf{d}}
\newcommand{\ebf}{\mathbf{e}}
\newcommand{\fbf}{\mathbf{f}}
\newcommand{\gbf}{\mathbf{g}}
\newcommand{\hbf}{\mathbf{h}}
\newcommand{\ibf}{\mathbf{i}}
\newcommand{\jbf}{\mathbf{j}}
\newcommand{\kbf}{\mathbf{k}}
\newcommand{\lbf}{\boldsymbol{\ell}}
\newcommand{\mbf}{\mathbf{m}}
\newcommand{\nbf}{\mathbf{n}}
\newcommand{\pbf}{\mathbf{p}}
\newcommand{\qbf}{\mathbf{q}}
\newcommand{\rbf}{\mathbf{r}}
\newcommand{\sbf}{\mathbf{s}}
\newcommand{\tbf}{\mathbf{t}}
\newcommand{\ubf}{\mathbf{u}}
\newcommand{\vbf}{\mathbf{v}}
\newcommand{\wbf}{\mathbf{w}}
\newcommand{\xbf}{\mathbf{x}}
\newcommand{\ybf}{\mathbf{y}}
\newcommand{\zbf}{\mathbf{z}}
\newcommand{\Fbf}{\mathbf{F}}
\newcommand{\Ibf}{\mathbf{I}}
\newcommand{\Jbf}{\mathbf{J}}
\newcommand{\Mbf}{\mathbf{M}}
\newcommand{\Tbf}{\mathbf{T}}
\newcommand{\Vbf}{\mathbf{V}}
\newcommand{\sth}{\text{s}_\theta}
\newcommand{\cth}{\text{c}_\theta}
\newcommand{\sph}{\text{s}_\phi}
\newcommand{\cph}{\text{c}_\phi}
\newcommand{\sps}{\text{s}_\psi}
\newcommand{\cps}{\text{c}_\psi}
\newcommand{\deltabf}{\boldsymbol{\delta}}
\newcommand{\omegabf}{\boldsymbol{\omega}}
\newcommand{\Omegabf}{\boldsymbol{\Omega}}
\newcommand{\Taubf}{\mathbf{\mathcal{T}}}
\newcommand{\zetabf}{\boldsymbol{\zeta}}
\newcommand{\zerobf}{\mathbf{0}}
\newcommand{\taubf}{\boldsymbol{\tau}}
\newcommand{\nubf}{\boldsymbol{\nu}}
\newcommand{\mubf}{\boldsymbol{\mu}}
\newcommand{\epsilonbf}{\boldsymbol{\epsilon}}
\newcommand{\ellbf}{\boldsymbol{\ell}}
\newcommand{\degrees}[1]{{#1}^\circ}
\newcommand{\cross}{\times}
\newcommand{\oneyear}{\rule{10mm}{3mm}}
\newcommand{\twoyears}{\multicolumn{2}{|c|}{\rule{26mm}{3mm}}}
\newcommand{\threeyears}{\multicolumn{3}{|c|}{\rule{44mm}{3mm}}}
\newcommand{\ith}{$i^{\text{th}}~$}
\newtheorem{theorem}{Theorem}[section]
\newtheorem{corollary}[theorem]{Corollary}
\newtheorem{definition}[theorem]{Definition}
\newtheorem{lemma}[theorem]{Lemma}
\newtheorem{property}[theorem]{Property}
\def\proof{\hspace{1em}{\it Proof: }}
\def\endproof{\hspace*{\fill}~$\blacksquare$\par\endtrivlist\unskip}
\newcommand{\subsubsubsection}[1]{{\par\noindent\it #1:}}
\def\proof{\hspace{1em}{\it Proof: }}
\def\endproof{\hspace*{\fill}~$\blacksquare$\par\endtrivlist\unskip}
\DeclareMathOperator{\sign}{sign}
\DeclareMathOperator{\sinc}{sinc}
\DeclareMathOperator{\atan2}{atan2}
\DeclareMathOperator{\asin}{asin}
\hyphenation{pseudo-range}

% define greybox:  for text in a grey box use \greybox{stuff}.
\long\def\greybox#1{%
    \newbox\contentbox%
    \newbox\bkgdbox%
    \setbox\contentbox\hbox to \hsize{%
        \vtop{
            \kern\columnsep
            \hbox to \hsize{%
                \kern\columnsep%
                \advance\hsize by -2\columnsep%
                \setlength{\textwidth}{\hsize}%
                \vbox{
                    \parskip=\baselineskip
                    \parindent=0bp
                    #1
                }%
                \kern\columnsep%
            }%
            \kern\columnsep%
        }%
    }%
    \setbox\bkgdbox\vbox{
        \pdfliteral{0.85 0.85 0.85 rg}
        \hrule width  \wd\contentbox %
               height \ht\contentbox %
               depth  \dp\contentbox
        \pdfliteral{0 0 0 rg}
    }%
    \wd\bkgdbox=0bp%
    \vbox{\hbox to \hsize{\box\bkgdbox\box\contentbox}}%
    \vskip\baselineskip%
}
% end definition of greybox
\newcommand{\Exp}{\ensuremath{\text{Exp}}}
\newcommand{\Log}[1]{\ensuremath{\text{Log}\left( #1 \right)}}

%%%%%%%%%%%%%%%%%%%%%%%%%%%%%%%%%%%%%%%%%%%%%%%%%%%%%%%%%
%                End Macros
%%%%%%%%%%%%%%%%%%%%%%%%%%%%%%%%%%%%%%%%%%%%%%%%%%%%%%%%%

%%%%%%%%%%%%%%%%%%%%%%%%%%%%%%%%%%%%%%%%%%%%%%%%%%%%%%%%%
%                James Jackson Macros
%%%%%%%%%%%%%%%%%%%%%%%%%%%%%%%%%%%%%%%%%%%%%%%%%%%%%%%%%

%\global\long\def\v{\mathbf{v}}%
%
%\global\long\def\u{\mathbf{u}}%
%
%\global\long\def\a{\mathbf{a}}%
%
%\global\long\def\b{\mathbf{b}}%
%
%\global\long\def\c{\mathbf{c}}%
%
%\global\long\def\w{\boldsymbol{\omega}}%
%
%\global\long\def\p{\mathbf{p}}%
%
%\global\long\def\t{\mathbf{t}}%
%
%\global\long\def\i{\mathbf{i}}%
%
%\global\long\def\j{\mathbf{j}}%
%
%\global\long\def\e{\mathbf{e}}%
%
%\global\long\def\k{\mathbf{k}}%
%
%\global\long\def\r{\mathbf{r}}%
%
%\global\long\def\d{\mathbf{d}}%
%
%\global\long\def\x{\mathbf{x}}%
%
%\global\long\def\y{\mathbf{y}}%
%
%\global\long\def\q{\mathbf{q}}%
%
%\global\long\def\qq{\Gamma}%
%
%\global\long\def\qr{\q_{r}}%
%
%\global\long\def\qd{\q_{d}}%
%
%\global\long\def\SO{\mathit{SO}}%
%
%\global\long\def\SE{\mathit{SE}}%
%
%\global\long\def\skew#1{\left\lfloor #1\right\rfloor _{\times}}%
%
%\global\long\def\norm#1{\left\Vert #1\right\Vert }%
%
%\global\long\def\grey#1{\textcolor{gray}{#1}}%
%
%\global\long\def\abs#1{\left|#1\right|}%
%
%\global\long\def\S{\mathcal{S}}%
%
%\global\long\def\dd{\boldsymbol{\delta}}%
%
%\global\long\def\Ad{\textrm{Ad}}%
\newcommand{\Ad}{\textnormal{Ad}}
%
%\global\long\def\SU{\mathit{SU}}%
%
%\global\long\def\R{\mathbb{R}}%
%
%\global\long\def\so{\mathfrak{so}}%
%
%\global\long\def\se{\mathfrak{se}}%
%
%\global\long\def\su{\mathfrak{su}}%
\newcommand{\so}{\mathfrak{so}}
\newcommand{\su}{\mathfrak{su}}
\newcommand{\se}{\mathfrak{se}}

%%%%%%%%%%%%%%%%%%%%%%%%%%%%%%%%%%%%%%%%%%%%%%%%%%%%%%%%%
%                End James Jackson Macros
%%%%%%%%%%%%%%%%%%%%%%%%%%%%%%%%%%%%%%%%%%%%%%%%%%%%%%%%%


%%%%%%%%%%%%%%%%%%%%%%%%%%%%%%%%%%%%%%%%%%%%%%%%%%%%%%%%%
%                Begin Formatting
%%%%%%%%%%%%%%%%%%%%%%%%%%%%%%%%%%%%%%%%%%%%%%%%%%%%%%%%%

%%% Commented these lines out for PUP
%\oddsidemargin 0.0in \evensidemargin 0.0in \marginparwidth 0pt
%\marginparsep 0pt \topmargin 0pt \headsep 0.25in \footskip 0.4in
%\textheight 8.5in \textwidth 6.5in \hoffset 0pt \voffset 0pt
%\paperheight 11in \paperwidth 8.5in \headheight 14pt
%\renewcommand{\baselinestretch}{1.2}

%\pagestyle{fancy}
%\renewcommand{\headrulewidth}{0pt}
%\renewcommand{\headsep}{30pt}
%\renewcommand{\chaptermark}[1]{\markboth{#1}{}}
%%\renewcommand{\sectionmark}[1]{\markright{\thesection\ #1}}
%\lhead[\fancyplain{}{\textsf{\bfseries\small\thepage}}]{\fancyplain{}{\textsf{\small\leftmark}}}
%%\rhead[\fancyplain{}{\bfseries\small\rightmark}]{\fancyplain{}{\bfseries\small\thepage}}
%\rhead[\fancyplain{}{\textsf{\small{Chapter}}~\textsf{\small\thechapter}}]{\fancyplain{}{\textsf{\bfseries\small\thepage}}}
%\cfoot{}
%
%\makeatletter
%\newenvironment{hw}{%
%  \renewcommand{\labelenumi}{\thechapter.\theenumi}%
%  \renewcommand{\p@enumi}{\thechapter.}%
%     \begin{enumerate}}{\end{enumerate}}
%\makeatother

%%%%%%%%%%%%%%%%%%%%%%%%%%%%%%%%%%%%%%%%%%%%%%%%%%%%%%%%%
%                End Formatting
%%%%%%%%%%%%%%%%%%%%%%%%%%%%%%%%%%%%%%%%%%%%%%%%%%%%%%%%%
